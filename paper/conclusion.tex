This paper aims to quantify the causal impact of iBuyers towards the US housing market price level and liquidity. By using a system of simultaneous equations and deriving their linear reduced form, this paper accounts for the endogeneity between TOM and price. These two dependent variables are then regressed against three sets of covariates that covers socioeconomic and housing characteristics, as well as demand and supply shifters. Additionally, this paper also incorporates the difference-in-difference approach to eliminate time-invariant unobserved heterogeneity that exists in the county level.  

All three difference-in-difference specifications result show that iBuyers do not have any significant impact towards the median listing price or TOM of the housing market. However, there exists limitations within this paper that originates from the lack of available data. One such limitation is the breach of the parallel trend assumption, originating from the lack of observations on key years. Another limitation is the sample selection bias that comes from iBuyers making decisions on which market they should enter. One possible solution is by adding a selection equation in addition to the simultaneous equation of housing prices and TOM.  

Future studied on this topic should consider gathering longer-term trends in iBuyer dynamics and perhaps focus on conducting studies on the individual property level. Incorporating older data, optimally beginning in 2013, would also improve results. Additionally, novel techniques such as using synthetic controls groups could mitigate the limitations of the current DD model.  

Ultimately, it is unclear whether this new trend in the real estate market is beneficial to the affected markets or just backed by shallow promises. Regardless, this trend will continue, and if this phenomenon is not better understood, the instant gratification could turn into regret in no time. 
