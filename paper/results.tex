This paper estimates the DD model specification in three different ways: without the covariates, with covariates including the number of iBuyers and with covariates but excluding the number of iBuyers. The results are shown in Table ~\ref{dd_res}.  

In the first model, TOM and log(price) are only regressed against the time dummy variable, treatment dummy variable and their interaction term. Results show that holding all else constant, an iBuyer in the market reduces the median listing price by 1.4\% and TOM by 3.7 days. However, these coefficient estimates are not statistically significant. This paper argues that the model without covariates suffers from an omitted variable bias. Thus, this paper supplements the analysis by incorporating covariates into the other two models.  

Looking at the estimates of the second and third models, the presence of an iBuyer increases the median listing price, holding all else constant, by 0.8 and 1\%, respectively. Given that housing prices range from $100,000 to $900,000, this small percentage might be significant in dollar terms.  However, both models show that this effect on price is not statistically significant. Similarly, the presence of an iBuyer only reduces TOM by 3.6 to 4.3 days. This is arguably insignificant when compared against the average TOM of 78.5 days. 

This paper proceeds to interpret the estimated coefficients by evaluating their sign, magnitude and statistical significance. 
