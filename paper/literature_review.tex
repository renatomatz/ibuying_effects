Real estate is one of the least liquid asset, requiring a considerable amount of time listed in the market before it is sold. This has led to the emergence of real estate brokerage services that aim to reduce market frictions by connecting buyers and sellers, and providing legal expertise. The literature is unclear on the effect of brokerage firms on housing market liquidity, as measured by time on market (TOM). Sirmans, Turnbull and Benjamin (1991) \cite{Sirman} found that larger firms were able to sell houses faster than their smaller counterparts. Yet, Yang and Yavas (1995) \cite{Yavas} on two separate studies found that the size does not have any significant effect on TOM.  

However, the literature suggests that TOM is endogenous, as it is affected by listing price and vice versa.  Miller (1978) \cite{Miller} treated TOM as a regression in explaining price. Miller finds that if time on the market is longer, the observed equilibrium price would be lower compared to the initial listing price. In contrast, Belkin et al. (1976) \cite{Belkin} considered TOM as a regressand. The researchers find the listing price is high, this would increase TOM. Wheaton (1990) \cite{Wheaton} as well as Krainer and Leroy (2002) \cite{Krainer} use a search-theoretic model to argue that price and TOM depend simultaneously on the likelihood of a sale made. Essentially, there is a positive correlation between prices and TOM. This paper accounts for this endogeneity and uses a simultaneous system to estimate the impact of iBuyer on the price and TOM of the housing market.   

In modeling the housing market, existing literature commonly treats houses as composite goods. This suggests that property price is jointly determined by consumer’s evaluation of the value of the property's observable attributes and producer’s offering price. Papers that models housing demand and supply in the micro-level focus on individual house characteristics, such as number of rooms, bathrooms, size, etc. However, another critical feature of houses as an economic good is the considerable cost associated with choosing another dwelling unit. Beyond just the costs of actual moving, there are search costs from transferring property possession and broker service fee. It is thus necessary to also account for quality of public services or accessibility of employment specific to the county of the house’s location. 
