Real estate is one of the least liquid asset, requiring a considerable amount of time listed in the market before it is sold. This has led to the emergence of real estate brokerage services that aim to reduce market frictions by connecting buyers and sellers, and providing legal expertise. The literature is unclear on the effect of brokerage firms on housing market liquidity, as measured by time on market (TOM). Sirmans, Turnbull and Benjamin (1991) \cite{Sirman} found that larger firms were able to sell houses faster than their smaller counterparts. Yet, Yang and Yavas (1995) \cite{Yavas} on two separate studies found that the size does not have any significant effect on TOM.  

Most brokers try to market themselves as having more information that allows them to sell homes at a faster and higher price than their competitors. The emergence of iBuyers with their algorithms and access to widespread data begs the question as to how this innovation would alter the US housing market. As far as we know, this is the first paper that aims to quantify the effect of iBuyers towards the TOM in the US housing market.  

However, TOM is endogenous; it is also affected by the listing price and vice versa.  Miller (1978) \cite{Miller} treated TOM as a regressor in explaining price. If time on the market is longer, the observed equilibrium price would be lower compared to the initial listing price. In the contrary, Belkin et al. (1976) \cite{Belkin} considered TOM as a regressand. If the listing price is high, this would increase TOM. Wheaton (1990) \cite{Wheaton} and Krainer and Leroy (2002) \cite{Krainer} argues by using a search-theoretic models that price and TOM depend simultaneously on the likelihood of a sale made. Essentially, there is a positive correlation between prices and TOM. This paper accounts for this endogeneity and uses a simultaneous system to estimate the impact of iBuyer on the price and TOM of the housing market.   

In modeling the housing market, existing literature commonly treats houses as composite goods. Its price is jointly determined by consumer’s evaluation of the value of its observable attributes and producer’s offering price. Papers that models housing demand and supply in the micro-level focuses on individual house characteristics, such as number of rooms, bathrooms, size, etc. However, another critical feature of houses as an economic good is the considerable cost associated with choosing another dwelling unit – there are search costs, cost of moving household possession and broker service fee. It is thus necessary to also account for quality of public services or accessibility of employment specific to the county of the house’s location. 

Using aggregate county-level data, this paper models the median listing price and TOM as a function of the average housing characteristics, amenities, demand and supply shifters. This paper will first analyze the datasets and specifies the price-TOM simultaneous model. It will then proceed to present and interpret the results, discussing the implications.   
