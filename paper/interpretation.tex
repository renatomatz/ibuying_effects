\subsection{Dependent Variable Analysis}
The first model is a single DD treatment-effect estimation that only includes five variables: the dependent variable (either TOM or log(price)), the treatment variable (Z\_it), the time variable (T\_it), and its interaction term. This model requires the assumption that only time-invariant unobserved heterogeneity is distorting the effect of iBuyers on TOM and house prices. The results indicate that the presence of iBuyers in the county reduces the median listing price by 1.4\%. Additionally, the results suggest that, on average, the presence of iBuyers in the county reduced the TOM of house by 3.7 days.  

This result aligns with expected reactions of the estate market market to the presence of iBuyers. The efforts of iBuyers in estimating the value of homes with their algorithmic pricing should reduce uncertainty and friction in the housing market. This would increase the number of houses transacted, which would in turn improve the market’s liquidity and transparency. The increase in volume and scale of homes listed would decrease the market price. At the same time, this increase would allow iBuyers to drop their fees. This, the decrease in price and TOM in counties with iBuyers is economically significant.  

However, the scale of this effect is small – given that the average time on market is 78.5 days (as seen in Table~\ref{general_stats}), a 3.7 day decrease in TOM is arguably insignificant. The percentage reduction in price is also small. However, as the average listing price in all counties is \$274,000, this decrease is translated to be around \$2,740 in absolute terms. 

Still, the changes in TOM and housing price are not statistically significant in the context of this first model. As seen in Table~\ref{dd_res}, the coefficient for housing price and TOM has a p-value of 0.841 and 0.212 respectively. This shows us that comparatively, the treatment effect of listed prices is less statistically significant than its effect on TOM.

However, as mentioned in Section~\ref{empirical_framework}, there exists the issue of sample selection combined with the diverging trend in the housing price index of control groups. Essentially, iBuyers select market with higher growth trends in the housing prices. This unaccounted selection of cities introduce an upward biased in price differences. In other words, because iBuyers are more likely to select cities with the highest potential of growth in listing price, resulting estimates underestimate the effects of iBuyers on prices. Little can be said about the time on market due unavailable data on the historical trend of median time on market prior to 2016. However, under the assumption that the housing price index is a good proxy for trends in TOM, estimates for the effect of iBuyers toward TOM is also underestimated.  

\subsection{Independent Variable Analysis}
This subsection accounts for changes in covariates that affect the changes in TOM and housing prices. As shown in Table~\ref{dd_res}, most coefficient of the covariates are statistically significant at the 10 \% confidence level.  

Most coefficients exhibit signs which are consistent with theory. For example, an increase in homeowner’s vacancy rate decreases the median listing price. When the market is characterized by vacant homes, prospective buyers have more bargaining power as this almost signifies an oversupply. This is further reinforced by the positive correlation between homeowner’s vacancy rate towards TOM. An increase in homeowner’s vacancy rate by one standard deviation increases the TOM by 3.84 days. Another example is average household size. An increase in the number of people in a household by 1 standard deviation increase the price level by 34.4\%. This aligns with theory, as large families or households require more spacious (and thus expensive) houses. 

However, some estimates have the opposite or even no relation to what we expect. For example, this model incorporated the median household income and population number as demand shifters that may affect housing prices or liquidity. However, the model estimates these variables to be 0. This means holding all else constant, an increase in median household income and population number affect neither housing price nor TOM. After further analysis, this paper found that there is little significant growth in population from 2016 to 2019. This is not the case with median household income, which increased from and average of \~\$58,000 to \~\$65,000 per capita. There are two scenarios that explain this result: first, median household income do not affect housing price level at all, which is highly unlikely. Second, household and per-capita incomes might still affect price level, but just on a very small economic level. It is important to note that this estimate is supported by a high t-statistics, and thus high statistical significance.

Another specification showing the opposite sign to what we expect is the median rooms. From the DD regression results, an increase in one standard deviation in median rooms reduces price by 36.4\%, holding all else constant. Higher prices in response to more rooms per listed house may make sense when considering individual home purchases. However, as this dataset is is the county-level, the average rooms per property might just incorporate unobserved effects of rural-urban separation. Cities are known for their high housing prices, yet the average rooms tend to be smaller than that of their rural counterparts. Additionally, when a person chooses to reside in a specific county, they are limited by their housing options in that location, which usually share similar characteristics. Ultimately, adding more urban/rural indicators could prevent this unexpected result from happening in future work.

It is important to point out that the number of both top or local iBuyers in the county had little statistical significance towards both affecting housing prices or TOM. With regards to housing prices, holding all else constant, an increase in the number of iBuyers in the county on average increases the median listing prices by 6.2\%. This is contradictory to the argument that iBuyers reduce friction in the market which in turn, improves liquidity and increase the volume of transacted house. However, there are two points to be made. First, this assumes that changes in housing liquidity and volume are extremely responsive to the entrance of iBuyers into the market. However, this assumption does not seem to conform to theory. The assumption disregards the fact that iBuyers aim to make a 5-7\% margin on each house they buy. If there exists many top iBuyers within a market, these large firms could have market power to significantly alter median listing prices. A similar argument can be made for the effect of the number of iBuyers on TOM. However, it is important to note that the number of iBuyers as covariates for TOM and price are not statistically significant, as they have high p-values. Still, future work could be done to test this hypothesis by re-estimating this model using new data on TOM and prices and see whether the entrance of iBuyers into a market would significantly impact the market's housing dynamics.

This raises the question of how the model would be affected when excluding the number of iBuyers, which is the third DD model specification of this paper. As observed in Table~\ref{dd_res}, there is almost no change in the coefficient estimates on all covariates. This reinforces the initial hypothesis stated in from the first DD model that iBuyers are not observed to have a significant impact on both housing prices and TOM.  
