Land is and has been one of the strongest storages of value available in modern society. As is evident from the 2008 Financial Crisis, land, and more specifically real estate, has the scale to significantly impact the national economy. More recently in the 2020 COVID-19, real estate has become a safe haven to investors and individuals seeking stability amidst a volatile market. Still, one issue that has limited the real estate market for years is its lack of liquidity, which affects especially individuals whose purchase marks a significant portion of their wealth.  

With the objective of tackling this issue, and backed by massive amounts of capital and data, Instant Buyers (commonly referred to as iBuyers) have grown to become prevalent investors in residential real estate. As their name suggests, IBuyers differ from traditional residential real estate investors as they reduce the time from offer to purchase to often less than a week. IBuyers leverage price data and statistical models to evaluate properties in bulk and without the need of a physical evaluator. Their hope is to buy underpriced properties quickly at a discount and hopefully resell it at competitive prices within a short amount of time. 

Since the success of Opendoor, the first major iBuyer, in 2014, several other firms have joined the market for quick buying and selling of residential real estate. While most firms operate at a local level, others, such as real estate listing giant Zillow, have set up operations throughout the United States and brought with them billions of dollars in venture capital. While these companies appeal to individuals for their quick, all-cash offers, they often try advertising themselves as generators of higher market liquidity and fairer prices.  

At their best, iBuyers back their claims with annecdotal evidence rather than numbers, and at the worst, IBuers could have a detrimental effect on markets. Though companies tout their qualities as home buyers, the true effect on a region's real estate market depends on their qualities as sellers. Given enough market share in a given area, iBuyers can create a local monopoly if they own a significant portion of available homes. If this is indeed an iBuyer's ultimate strategy, we should be seeing higher prices and a possible increase in the time needed for a property to be sold.

In this study, we aim to test this claim that iBuyers improve market liquidity and affect market prices in severl US counties. We will explore whether the iBuyer's claim is in fact true, and whether their presence in a county can in fact affect property prices and reduce the time needed to sell a house in their acting region.
